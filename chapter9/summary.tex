This thesis has focused on the study of low-mass stars and their companions,
whether these companions are planets, brown dwarfs, or themselves other M dwarfs.
While this work has been able to advance the study of all three of these classes of
objects, there is more that can be done in the future as new data are collected and new
instruments are built at new facilities.
Let us consider each of these classes of objects in turn.

In Chapter \ref{chap:trends}, I developed a method to measure the occurrence rate of giant
planets out to 20 AU through a combination of high-contrast direct imaging and 
detections of long-term RV trends.
The planets inferred in this work form a unique region of parameter space as yet 
inaccessible to searches via other methods:
they are too far from their host stars to be detectable through transit searches,
but too near and too faint to be detectable through direct imaging campaigns alone.
The method developed here is directly applicable to higher-mass host stars, and the
process can be extended to directly measure the occurrence rate of planets around K and G
stars.
With the larger number of G and K dwarfs observed in RV surveys and the longer observational
time baseline, it is possible that we will be able to measure the occurrence rate of 
giant planets around these stars to an even higher precision.
Some of this work is ongoing, while the \textit{Gaia} telescope will provide more  information about Jupiter-like planets in the solar neighborhood in the second half of
this decade.
In Chapter \ref{chap:k2} I presented a method to understand the stellar and planetary parameters of candidate systems uncovered by \KT, a process that can also be applied to the systems detected by \textit{WFIRST} (Chapter \ref{chap:wfirst}. 
\KT\ will enable us to better understand the population of planets orbiting M dwarfs 
in the solar neighborhood. 
By the end of the mission, \KT\ will search for transit signals
around as many as 50,000 M dwarfs, while the original \kep\ mission observed 5,000.
Moreover, \KT\ will observe mid-M dwarfs, where the original \kep\ mission largely eschewed
stars later than M1.
Both \KT\ and \textit{WFIRST} will target higher mass stars in vastly different galactic
environments: \KT\ has targeted young clusters, stars in and well out of the galactic 
plane, and stars at varied galactocentric distances.
\textit{WFIRST} will similarly be able to detect transits around solar-type stars
more than 2 kiloparsec away in the direction of the galactic bulge. 
Following the galactic metallicity gradient of \citep{Rolleston00}, in which the 
metallicity of stars decreases at $0.07 \pm 0.01$ dex kpc$^{-1}$, we might expect
these stars to host Jupiter-sized planets 50\% more frequently than stars
near the Sun.
\textit{WFIRST} will enable us to test theories of planet formation and the relation
between stellar metallicity and planet formation with unprecendented detail.

Next, we turn our attention to brown dwarfs.
In Chapters \ref{chap:lhs1} and \ref{chap:lhsspitz} I presented LHS\,6343\,C, the
brown dwarf with the most precise radius measurement and the only brown dwarf with a
direct mass, radius, and luminosity measurement.
LHS\,6343\,C can provide a single test of brown dwarf models, but more similar
brown dwarfs are required.
Fortunately, more similar brown dwarfs are being discovered. 
In 2016, \citet{Bayliss16} presented radial velocity data on EPIC\,201702477\,b,
one of the objects of interest characterized in Chapter \ref{chap:k2}. 
I was not able to confirm or rule out the planetary nature of this system;
with radial velocities, these authors were able to confirm the system as a transiting
brown dwarf with a period of 40.74 days.
Similarly, other work has discovered a transiting brown dwarf in the 3 Gyr Ruprecht 147
cluster (Curtis et al.\ \textit{in prep}). 
As more systems like these are discovered, especially systems with known ages, 
we will be able to fill the brown dwarf mass-radius diagram and connect the transiting
brown dwarf population to the field brown dwarf population, for which direct measurements
of masses and radii are impossible.


Finally, we consider M+M binaries. 
Throughout the M dwarf spectral class, more objects with direct mass measurements are
needed.
Nearly all mass measurements come from eclipsing binaries in short periods, which may
cause inflated radii due to magnetic activity \citep{Chabrier07, Jackson09}.
\KT\ and \textit{WFIRST} will enable the discovery of more M dwarf eclipsing binaries 
on longer periods, which are less likely to be inflated.
In Chapter \ref{chap:ttvs}, I describe a method to measure masses of single stars
with transiting planets by combining RV and TTV observations of the planetary system,
avoiding the potential complications of binary stars entirely.

Very young stars rotate rapidly and are photometrically very active, complicating both
the detection of transiting planets and the precision RVs required to confirm them 
directly.
The problems with stellar models at these ages are even worse, with only a handful of
M dwarfs younger than 100 Myr having directly measured masses.
In Chapter \ref{chap:Mbinaries} I presented GJ\,3305\,AB, a binary M+M system in the
$\beta$ Pictoris young moving group. 
I measured the mass of both components, comparing them to stellar models.
This is only one binary system of the dozens I am monitoring astrometrically and through
RV observations with collaborators. There are more than 20 more stars in more than 
10 systems with orbits that have closed or will close in the next few years in the same
mass and age range.
As these orbits close, we will develop a population of stars with measured masses to
compare against models, enabling the development of the next generation of stellar models.

Of course, much of the future work will rely on future instruments and future telescopes
both in the ground and in space. 
In this thesis I discuss the ongoing \KT\ and future \textit{WFIRST} missions.
beyond these, the study of M dwarfs can look forward to contributions from future 
transit search missions like \textit{TESS} \citep{Ricker14} and \textit{PLATO} \citep{Catala10}, as well as the 
development and construction of the next generation of 30-meter class telescopes.
Moreover, RV instruments that are optimized for observations in the near-IR, such as
\textit{iLocator} \citep{Crepp14} and \textit{MAROON-X} \citep{Seifahrt16}, will enable
later, more distant M dwarfs to be more easily targeted in RV surveys.
While the equations of stellar structure force M dwarfs to be faint, their future
is very bright indeed.