%%%%%%%%%%%%
%% Please rename this main.tex file and the output PDF to
%% [lastname_firstname_graduationyear]
%% before submission.
%%%%%%%%%%%%

\documentclass[12pt]{caltech_thesis}
\usepackage[hyphens]{url}
\usepackage{lipsum}
\usepackage{graphicx}
\usepackage{xcolor}

%\usepackage{todonotes}

%% Tentative: newtx for better-looking Times
\usepackage[utf8]{inputenc}
\usepackage[T1]{fontenc}
\usepackage{newtxtext,newtxmath}

% Must use biblatex to produce the Published Contents and Contributions, per-chapter bibliography (if desired), etc.
\usepackage[
    backend=biber,natbib,
    %backend=natbib,
    % IMPORTANT: load a style suitable for your discipline
    style=authoryear-comp,
    doi=False,
    bibstyle=apj
]{biblatex}
\renewcommand*{\nameyeardelim}{\addspace}
%\usepackage{natbib}
%\bibliographystyle{apj}
\newcommand{\todo}[3]{{\color{#2} \emph{#1} TO DO: #3}}
\newcommand{\btmtodo}[1]{\todo{BEN}{red}{#1}}

\renewcommand{\d}[1]{\ensuremath{\operatorname{d}\!{#1}}}


% Name of your .bib file(s)
\addbibresource{exopapers.bib}
\addbibresource{example.bib}

\begin{document}

% Do remember to remove the square bracket!
\title{Low-Mass Stars and Their Companions}
\author{Benjamin Tyler Montet}

\degreeaward{Doctor of Philosophy}                 % Degree to be awarded
\university{California Institute of Technology}    % Institution name
\address{Pasadena, California}                     % Institution address
\unilogo{caltechseal2.png}                                 % Institution logo
\copyyear{2017}  % Year (of graduation) on diploma
\defenddate{July 18, 2016}          % Date of defense

\orcid{0000-0001-7516-8308}

%% IMPORTANT: Select ONE of the rights statement below.
%\rightsstatement{All rights reserved\todo[size=\footnotesize]{Choose one from the choices in the source code!! And delete this \texttt{todo} when you're done that. :-)}}
 \rightsstatement{All rights reserved except where otherwise noted}
% \rightsstatement{Some rights reserved. This thesis is distributed under a [name license, e.g., ``Creative Commons Attribution-NonCommercial-ShareAlike License'']}

%%  If you'd like to remove the Caltech logo from your title page, simply remove the "[logo]" text from the maketitle command
\maketitle[logo]
%\maketitle

\begin{acknowledgements} 	 
   People to thank: John + group. Brendan Bowler, Justin, Luan, 
   Hogg, DFM, Ruth. 
   Ian, Dawn, Ryan, Jieun, Aaron and Jason
   Cahill department and traditions - see Sirio's.
   Antonija + Mislav, Trevor, Allison, Yi. Matt, Sirio, Gwen and Drew, Ryan.
   Committee, Patrick and Anu
   Laura
   Parents
   
\end{acknowledgements}

\begin{abstract}
   [This abstract must provide a succinct and informative condensation of your work. Candidates are welcome to prepare a lengthier abstract for inclusion in the dissertation, and provide a shorter one in the CaltechTHESIS record.]
\end{abstract}

%% Uncomment the `iknowhattodo' option to dismiss the instruction in the PDF.
%\begin{publishedcontent}%[iknowwhattodo]
% List your publications and contributions here.
%\nocite{Cahn:etal:2015}
%\end{publishedcontent}

\tableofcontents
\listoffigures
\listoftables
%\printnomenclature

\mainmatter

\chapter{Introduction}

\section{Why Ben gets a PhD}

PhDs are for awesome people \cite{I got's me one}. The awesome of Ben ($A_{ben}$) is defined by

\begin{align}
A_{ben} $= A_0^{\infty}
\end{align}

Where $A_0$ is normalized to Tony Stark and expressed in SI units of Fonzarelli's \cite{https://en.wikipedia.org/wiki/Fonzie}.

Thus, Ben get's a PhD! QED


\section{This is a Section}
\lipsum[1-2]

\begin{figure}[hbt!]
\centering
\includegraphics[width=.3\textwidth]{caltech.png}
\caption{This is a figure}\label{fig:logo}
\index{Example Figure}
\end{figure}

\subsection{This is a subsection}

\begin{table}[hbt!]
\centering
\begin{tabular}{ll}
\hline
Area & Count\\
\hline
North & 100\\
South & 200\\
East & 80\\
West & 140\\
\hline
\end{tabular}
\caption{This is a table}\label{tab:sample}
\index{tables}
\end{table}

\lipsum[3] \nomenclature{Asteroid}{A very small planet ranging from 1,000 km to less than one km in diameter. Asteroids are found commonly around other larger planets}

\lipsum[4-5] 

Here's an endnote.\endnote{Endnotes are notes that you can use to explain text in a document.}

\section{This is Another Section}
\lipsum[6-7] 

\chapter{This is the Second Chapter}
%\begin{refsection}
%If you'd like to have separate bibliographies at the end of each chapter, %put a \verb|refsection| around the material of each chapter, then cite as usual -- e.g.~\citep{GMP81,Ful83}. Then do a \verb|\printbibliography| just before the \verb|refsection| ends. \index{bibliography!by chapter}

%\printbibliography[heading=subbibliography]
%\end{refsection}


\chapter{This is the Third Chapter}

%\publishedas{Cahn:etal:2015}

[You can have chapters that were published as part of your thesis. The text style of the body should be single column, as it was submitted to the publisher, not formatted as the publisher did.]

\chapter{This is the Fourth Chapter}
\chapter{This is the Fifth Chapter}
\chapter{This is the Sixth Chapter}
\chapter{This is the Seventh Chapter}
\chapter{This is the Eighth Chapter}

\printbibliography[heading=bibintoc]
%\bibliography{exopapers}

%\appendix

%\chapter{Questionnaire}
%\chapter{Consent Form}

%\printindex

%\theendnotes

%% Pocket materials at the VERY END of thesis
%\pocketmaterial
%\extrachapter{Pocket Material: Map of Case Study Solar Systems} 


\end{document}

